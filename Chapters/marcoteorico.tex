\chapter{Marco Te\'{o}rico}
\label{ch:marco}
\section{Inteligencia Artificial}
La Inteligencia Artificial (IA) como disciplina formal  ha sido estudiada por más de 50 años. En sus principios, uno de los principales objetivos de esta disciplina era determinar si una computadora era capaz de “pensar", esta pregunta es igualmente abordada por Alan Turing\cite{turing1950computing} en su obra \emph{Computing Machinery and Intelligence}. La Inteligencia Artificial de manera simple puede ser definida como: el estudio de procesos mentales o intelectuales en t\'{e}rminos de procesos computacionales \cite{aimodern}. Sin embargo, esta definici\'{o}n varia dentro de la literatura \cite{haugeland1985,bellman1978,winston1992learning}. 

Como se puede apreciar, la Inteligencia Artificial tiene una relaci\'{o}n intr\'{i}nseca con la Psicolog\'{i}a, las Ciencias Cognitivas y, a su vez, con las Ciencias en Computaci\'{o}n. Sin embargo, la Inteligencia Artificial no es Psicolog\'{i}a, la distinci\'{o}n recae en que una de las mayores preocupaciones de la Inteligencia Artificial es comprehender de qu\'{e} manera es posible otorgarle a las computadoras una capacidad tan sofisticada como lo es actuar de forma “inteligente".
\section{C\'{o}mputo Cognitivo}
\section{Arquitecturas de Computadoras}
\section{Arquitecturas Cognitivas}
\section{Redes Bayesianas}
\section{Modelos de Markov Ocultos}
\section{Grafos de Factor}