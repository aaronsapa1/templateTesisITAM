\chapter{Introducción}
\label{ch:intro}

\section{Antecedentes}
El C\'{o}mputo Cognitivo es un nuevo paradigma de la computaci\'{o}n. Esta disciplina tiene como objetivo el desarrollar un mecanismo coherente inspirado en las capacidades de la mente humana. Se busca implementar una teor\'{i}a universal unificada de la mente humana. De esta forma, al querer estudiar la mente, se ve implicada la participaci\'{o} de la Neurociencia. El campo de la Computaci\'{o}n Cognitiva lleva los conceptos de la computaci\'{o}n a un nivel completamente nuevo \cite{cc}.

Uno de los temas m\'{a}s relacionados con este nuevo paradigma es descubrir de qu\'{e} forma ocurren los procesos cognitivos en los seres humanos. Este no es un problema trivial ya que incluso hoy en d\'{i}a se cuentan con diferentes teor\'{i}as. Este problema se extiende al C\'{o}mputo Cognitivo, pues hist\'{o}ricamente, diferentes campos han cuentan con diversos enfoques para estudiar el tema.

Por el lado de la Inteligencia Artificial, como rama de las Ciencias Cognitivas, cuenta con un enfoque a nivel de un sistema computacional. Esto significa que los avances de estudiar el C\'{o}mputo Cognitivo, por una parte, estar\'{a}n dirigidos a obtener nuevos sistemas de aprendizaje, arquitecturas de computadoras diferente a las tradicionales (t\'{o}mese por ejemplo las arquitecturas Von Neuman), sistemas capaces de analizar y sintetizar grandes cantidades de datos no estructurados, por mencionar algunos ejemplos.

Hoy en d\'{i}a, una de las principales empresas interesadas en el desarrollo de esta nueva ciencia, es IBM. Esta empresa es famosa por crear una computadora con la capacidad de ganarle a cualquier ser humano en un juego televisivo llamado “Jeopardy". Esta computadora fue llamada “Watson". En este juego se demostr\'{o} la capacidad de IBM de construir computadoras poderosas con capacidades sorprendentes. 

Ahora, IBM busca imitar las capacidades del cerebro ya que este es capaz de integrar procesamiento y memoria en un solo lugar, pesa menos de tres libras, ocupa un volumen de aproximadamente dos litros y utiliza menos energ\'{i}a que una bombilla de luz. De acuerdo con IBM, el cerebro es un sistema de aprendizaje capaz de ser modificado y tolerante a fallas, adem\'{a}s de ser extremadamente eficiente al momento de reconocer patrones. 

La Computaci\'{o}n Cognitiva hace referencia a un sistema con la capacidad de aprender a escala, razonar con prop\'{o}sito e interactuar con humanos de forma natural. En lugar de estar estrictamente programadas, los sistemas cognitivos aprenden y razonan de sus interacciones. Con base en modelos probabil\'{i}sticos, estos sistemas cuentan con la habilidad de generar hip\'{o}tesis, argumentos racionales y recomendaciones con base en una gran cantidad de datos no estructurados \cite{ibmcc}.

Como se mencion\'{o} anteriormente, IBM present\'{o} al mundo en Febrero del 2011 a \"Watson", actualmente es considerado como el primer sistema cognitivo y fue capaz de vencer a dos de los mejores jugadores de \"Jeopardy" del mundo. Esta fue la primera demostraci\'{o}n de lo que son capaces los sistemas cognitivos. De acuerdo con IBM, el nivel de \'{e}xito de los sistemas cognitivos ya no ser\'{a} medido con base en la prueba de Turing o la capacidad de la computadora de imitar capacidades humanas, sino por de manera m\'{a}s pr\'{a}ctica: oportunidades en nuevos mercados, retorno de la inversi\'{o}n, vidas salvadas \cite{ibmcc}.

Debido al gran crecimiento y beneficios que presentan los sistemas cognitivos, la necesidad de su implementaci\'{o}n y divulgaci\'{o}n se hace por medio de arquitecturas cognitivas. Una arquitectura cognitiva especifica la infraestructura en la cual se construye un sistema inteligente. En general, una arquitectura involucra ciertos elementos necesarios para la construcci\'{o}n de agentes cognitivos, como lo son: una arquitectura de almacenamiento (an\'{a}logo a las memorias de largo y corto plazo de la mente humana) en donde se almacenan las creencias, metas y conocimientos del agente; una representaci\'{o}n de los elementos contenidos en la memoria, es decir, una representaci\'{o}n del conocimiento y su organizaci\'{o}n; por \'{u}ltimo, una descripci\'{o}n de su proceso funcional (la forma en que adquiere conocimiento) que opera en las estructuras antes mencionadas \cite{langley2009cognitive}. Sin embargo, cabe aclarar que an\'{a}lo a una computadora, el conocimiento y las creencias de un agente cognitivo no se consideran como embebidas en la arquitectura. As\'{i} como diferentes programas de software pueden ser ejecutados en una misma computadora, distintas creencias y conocimiento pueden ser interpretados por una misma arquitectura.

Por otro lado, a diferencia de los sistemas expertos, una arquitectura se enfoca en ampliar el conocimiento que pueda adquirir por medio de diversas tareas y entornos; los sistemas expertos cuentan con un comportamiento que busca desarrollar un nivel de maestr\'{i}a en contextos reducidos y bien definidos.

Para ejemplificar los avances realizados en el desarrollo de arquitecturas cognitivas, se pueden mencionar diferentes marcos de trabajo los cuales comparten las caracter\'{i}isticas antes mencionadas. La principal raz\'{o}n por la cual se mencionan estos trabajos se debe a su popularidad. En primer lugar, como se ha mencionado anteriormente “Watson" de IBM es uno de los primero sistemas cognitivos y ha planteado el camino a seguir para las investigaciones futuras hoy en d\'{i}a ya se cuenta con un marco de trabajo que permite aprovechar las herramientas de “Watson" para sistemas expertos. En segundo lugar \emph{ACT-R} \cite{anderson1997act} es de las arquitecturas cognitivas m\'{a}s recientes y se enfoca principalmente en el modelado del comportamiento humano. En tercer lugar, \emph{Soar} \cite{soar1987} es una de las arquitecturas m\'{a}s importantes y populares debido a su enfoque, el cual consiste en que todas las tareas son formuladas como metas que, a su vez, estas metas ser organizan de forma jer\'{a}rquica. En cuarto lugar, \emph{ICARUS} \cite{langley1991design} se distingue debido a que almacena dos tipos distintos de conocimiento: conceptos, que describen clases de situaciones ambientales y habilidades que especifican c\'{o}mo cumplir con las metas establecidas. Por \'{ultimo}, \emph{PRODIGY} \cite{carbonell1991prodigy} que utiliza reglas de dominio y reglas de control para representar el conocimiento.

\section{Objetivos}

Esta tesis se enfoca en el desarrollo de las arquitecturas cognitivas, particularmente, en el marco de trabajo \emph{SOAR}, el cual es capaz de: trabajar sobre una gran variedad de diversas tareas; utiliza m\'{e}todos de resoluci\'{o}n de problemas y aprende de cada tarea realizada y analiza su propio desemp\~no. Gracias a que \emph{SOAR} es de libre c\'{o}digo, esta tesis implementa grafos de factor c\'{o}mo herramienta para lograr una implementaci\'{o}n unificada y al mismo tiempo contar con una gran diversidad al nivel de la arquitectura, es decir, que esta sea capaz de realizar un gran n\'{u}mero de actividades y desempe\~nar distintas tareas.

Adicionalmente, con base en el trabajo realizado por Roosenbloom se demuestra los beneficios que ofrece la implementaci\'{o}n de grafos de factor en comparaci\'{o}n con otros m\'{e}todos de resoluci\'{o}n de problemas. Debido a que los grafos de factor integran razonamiento simb\'{o}lico y probabil\'{i}stico, lo que genera un razonamiento m\'{a}s general bajo incertidumbre, permiten uniformidad a nivel de la implementaci\'{o}n en arquitecturas cognitivas como \emph{SOAR}\cite{rosenbloom2009towards}.

Por \'{u}ltimo, la intenci\'{o}n es contribuir en el desarrollo de las arquitecturas cognitivas, ya que representan una gran y novedosa herramienta \cite{langley1991design} con capacidades de resoluci\'{o}n de problemas a un nuevo nivel computacional. 

\section{Alcance}

Debido a la gran complejidad que representa la implementaci\'{o}n de una arquitectura cognitiva, este trabajo se limita a aumentar la capacidad de un solo marco de trabajo, en particular, la arquitectura \emph{SOAR}. Partiendo del art\'{i}culo de Roosenbloom, esta tesis implementa los grafos de factor para demostrar que representan una herramienta \'{u}til en este tipo de arquitecturas. 

Por otro lado, al tratarse de un marco de trabajo, la validaci\'{o}n de la implementaci\'{o}n consiste en medir el desempe\~no del sistema al realizar diferentes tareas de resoluci\'{o}n de problemas que, con anterioridad, \emph{SOAR} es capaz de desempe\~nar y este trabajo analiza la mejora o deterioro de la implementaci\'{o}n. 

\section{Justificaci\'{o}n}

La investigaci\'{o}n y desarrollo de las arquitecturas cognitivas es importante pues ayuda a fundamentar el objetivo central de la Inteligencia Artificial y las Ciencias Cognitivas. El objetivo es la craci\'{o}n y entendimiento de agentes sint\'{e}ticos que tengan las mismas capacidades cognitivas de los humanos. 

Sin embargo, aunque los campos de estudio entre la Psicolog\'{i}a y la Inteligencia Artificial se han distanciado en los \'{u}ltimos veinte a\~nos, la investigaci\'{o}n dentro de este paradigma cognitivo, permitir\'{a} ahondar en el entendimiento del funcionamiento y naturaleza de la mente humana. De la misma forma, se puede reconstruir las relaciones que se han perdido con la psicolog\'{i}a y sustentarlas con principios neurol\'{o}gicos para la creaci\'{o}n de nuevas teor\'{i}as de aprendizaje y modelos de representaci\'{o}n del conocimiento.

 No solo por beneficiar a las teor\'{a}s de aprendizaje se justifica el estudio en arquitecturas cognitivas, tambi\'{e}n es importante resaltar que la complejidad de los procesos cognitivos en el ser humano son dignas de estudio al ser un paradigma que representa grandes retos y oportunidades de desarrollo \cite{langley2006intelligent} .
 
 Adicionalmente, la creci\'{o}n de arquitecturas cognitivas como fundamentos para sistemas m\'{a}s complejos implica una mayor capacidad de craci\'{o}n de agentes inteligentes capaces de resolver problemas de manera eficiente, lo cual beneficia a la humanidad como una herramienta m\'{a}s de procesamiento y desarrollo tecnol\'{o}gico.
 
 Por \'{u}ltimo, las arquitecturas cognitivas conforman una “oposici\'{o}n" a los sistemas expertos, ya que estas arquitecturas ampl\'{i}an ampl\'{i}an su dominio del conocimiento con base en procesos funcionales que modifican su propio conocimiento, creencias y objetivos por medio de mecanismos de aprendizaje que dependen, en su mayor\'{i}a de de situaciones del entorno del agente. Mientras que los sistemas expertos, son orientados a logran una mayor eficiencia en actividades con l\'{i}mites bien definidos y reducen su dominio del conocimiento a problemas en espec\'{i}fico.
 
 \section{Metodolog\'{i}a}