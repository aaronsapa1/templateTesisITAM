\chapter{Introducción}
\label{ch:intro}

\section{Antecedentes}

La Inteligencia Artificial (IA) como disciplina formal  ha sido estudiada por más de 50 años. En sus principios, uno de los principales objetivos de esta disciplina era determinar si una computadora era capaz de “pensar", esta pregunta es igualmente abordada por Alan Turing\cite{turing1950computing} en su obra \emph{Computing Machinery and Intelligence}. La Inteligencia Artificial de manera simple puede ser definida como: el estudio de procesos mentales o intelectuales en t\'{e}rminos de procesos computacionales \cite{aimodern}. Sin embargo, esta definici\'{o}n varia dentro de la literatura \cite{haugeland1985,bellman1978,winston1992}. 

Como se puede apreciar, la Inteligencia Artificial tiene una relaci\'{o}n intr\'{i}nseca con la Psicolog\'{i}a, las Ciencias Cognitivas y, a su vez, con las Ciencias en Computaci\'{o}n. Sin embargo, la Inteligencia Artificial no es Psicolog\'{i}a, la distinci\'{o}n recae en que una de las mayores preocupaciones de la Inteligencia Artificial es comprehender de qu\'{e} manera es posible otorgarle a las computadoras una capacidad tan sofisticada como lo es actuar de forma “inteligente".

Ahora bien, de lado de las Ciencias Cognitivas, nace un nuevo paradigma de la computaci\'{i}on: Computaci\'{o}n Cognitiva. Esta disciplina tiene como objetivo el desarrollar un mecanismo coherente y unificado, inspirado en las capacidades de la mente humana. Se busca implementar una teor\'{i}a universal unificada de la mente humana. De esta forma, al querer estudiar la mente, se ve implicada la participaci\'{o} de la Neurociencia. El campo de la Computaci\'{o}n Cognitiva lleva los conceptos de la computaci\'{i}on a un nivel completamente nuevo \cite{cc}.

Hoy en d\'{i}a, una de las principales empresas interesadas en el desarrollo de esta nueva ciencia, es IBM. Esta empresa es famosa por crear una computadora con la capacidad de ganarle a cualquier ser humano en un juego televisivo llamado “Jeopardy". Esta computadora fue llamada “Watson". En este juego se demostr\'{o} la capacidad de IBM de construir computadoras poderosas con capacidades sorprendentes. 

Ahora, IBM busca imitar las capacidades del cerebro ya que este es capaz de integrar procesamiento y memoria en un solo lugar, pesa menos de tres libras, ocupa un volumen de aproximadamente dos litros y utiliza menos energ\'{i}a que una bombilla de luz. De acuerdo con IBM, el cerebro es un sistema de aprendizaje capaz de ser modificado y tolerante a fallas, adem\'{a}s de ser extremadamente eficiente al momento de reconocer patrones. 

La Computaci\'{o}n Cognitiva hace referencia a un sistema con la capacidad de aprender a escala, razonar con prop\'{o}sito e interactuar con humanos de forma natural. En lugar de estar estrictamente programadas, los sistemas cognitivos aprenden y razonan de sus interacciones. Con base en modelos probabil\'{i}sticos, estos sistemas cuentan con la habilidad de generar hip\'{o}tesis, argumentos racionales y recomendaciones con base en una gran cantidad de datos no estructurados \cite{ibmcc}.

Como se mencion\'{o} anteriormente, IBM present\'{o} al mundo en Febrero del 2011 a \"Watson", actualmente es considerado como el primer sistema cognitivo y fue capaz de vencer a dos de los mejores jugadores de \"Jeopardy" del mundo. Esta fue la primera demostraci\'{o}n de lo que son capaces los sistemas cognitivos. De acuerdo con IBM, el nivel de \'{e}xito de los sistemas cognitivos ya no ser\'{a} medido con base en la prueba de Turing o la capacidad de la computadora de imitar capacidades humanas, sino por de manera m\'{a}s pr\'{a}ctica: oportunidades en nuevos mercados, retorno de la inversi\'{o}n, vidas salvadas \cite{ibmcc}.

Debido al gran crecimiento y beneficios que presentan los sistemas cognitivos, la necesidad de su implementaci\'{o}n y divulgaci\'{o}n se hace por medio de arquitecturas cognitivas. Una arquitectura cognitiva especifica la infraestructura en la cual se construye un sistema inteligente. En general, una arquitectura involucra ciertos elementos necesarios para la construcci\'{o}n de agentes cognitivos, como lo son: una arquitectura de almacenamiento (an\'{a}logo a las memorias de largo y corto plazo de la mente humana) en donde se almacenan las creencias, metas y conocimientos del agente; una representaci\'{o}n de los elementos contenidos en la memoria, es decir, una representaci\'{o}n del conocimiento y su organizaci\'{o}n; por \'{u}ltimo, una descripci\'{o}n de su proceso funcional (la forma en que adquiere conocimiento) que opera en las estructuras antes mencionadas \cite{langley2009cognitive}. Sin embargo, cabe aclarar que an\'{a}lo a una computadora, el conocimiento y las creencias de un agente cognitivo no se consideran como embebidas en la arquitectura. As\'{i} como diferentes programas de software pueden ser ejecutados en una misma computadora, distintas creencias y conocimiento pueden ser interpretados por una misma arquitectura.

Por otro lado, a diferencia de los sistemas expertos, una arquitectura se enfoca en ampliar el conocimiento que pueda adquirir por medio de diversas tareas y entornos; los sistemas expertos cuentan con un comportamiento que busca desarrollar un nivel de maestr\'{i}a en contextos reducidos y bien definidos.

Para ejemplificar los avances realizados en el desarrollo de arquitecturas cognitivas, se pueden mencionar diferentes marcos de trabajo los cuales comparten las caracter\'{i}isticas antes mencionadas. La principal raz\'{o}n por la cual se mencionan estos trabajos se debe a su popularidad. En primer lugar, como se ha mencionado anterior mente “Watson" de IBM es uno de los primero sistemas cognitivos y ha planteado el camino a seguir para las investigaciones futuras. En segundo lugar \emph{ACT-R} \cite{anderson1997act} es de las arquitecturas cognitivas m\'{a}s recientes y se enfoca principalmente en el modelado del comportamiento humano. En tercer lugar, \emph{Soar} \cite{soar1987} es una de las arquitecturas m\'{a}s importantes y populares debido a su enfoque, el cual consiste en que todas las tareas son formuladas como metas que, a su vez, estas metas ser organizan de forma jer\'{a}rquica. En cuarto lugar, \emph{ICARUS} \cite{langley1991design} se distingue debido a que almacena dos tipos distintos de conocimiento: conceptos, que describen clases de situaciones ambientales y habilidades que especifican c\'{o}mo cumplir con las metas establecidas. Por \'{ultimo}, \emph{PRODIGY} \cite{carbonell1991prodigy} que utiliza reglas de dominio y reglas de control para representar el conocimiento.

